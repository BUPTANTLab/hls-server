\documentclass[table]{beamer}
%\documentclass[draft,table]{beamer}
%[]中可以使用handout、trancompress等参数

%指定beamer的模式与主题
\mode<presentation>
{
  \usetheme{Frankfurt}
%\usetheme{Boadilla}
\usecolortheme{default}
\usecolortheme{orchid}
\usecolortheme{whale}
\usefonttheme{professionalfonts}
}

%\usetheme{Madrid}
%这里还可以选择别的主题:Bergen, Boadilla, Madrid, AnnArbor, CambridgeUS, Pittsburgh, Rochester, Warsaw, ...
%有导航栏的Antibes, JuanLesPins, Montpellier, ...
%有内容的Berkeley, PaloAlto, Goettingen, Marburg, Hannover, ...
%有最小导航栏的Berlin, Ilmenau, Dresden, Darmstadt, Frankfurt, Singapore, Szeged, ...
%有章和节表单的Copenhagen, Luebeck, Malmoe, Warsaw, ...

%\usecolortheme{default}
%设置内部颜色主题(这些主题一般改变block里的颜色);这个主题一般选择动物来命名
%这里还可以选择别的颜色主题,如默认的和有特别目的的颜色主题default,structure,sidebartab,全颜色主题albatross,beetle,crane,dove,fly,seagull,wolverine,beaver

%\usecolortheme{orchid}
%设置外部颜色主题(这些主题一般改变title里的颜色);这个主题一般选择植物来命名
%这里还可以选择别的颜色主题,如默认的和有特别目的的颜色主题lily,orchid,rose

%\usecolortheme{whale}
%设置字体主题;这个主题一般选择海洋动物来命名
%这里还可以选择别的颜色主题,如默认的和有特别目的的颜色主题whale,seahorse,dolphin

%\usefonttheme{professionalfonts}
%类似的还可以定义structurebold,structuresmallcapsserif,professionalfonts


% 控制 beamer 的风格,可以根据自己的爱好修改
%\usepackage{beamerthemesplit} %使用 split 风格
%\usepackage{beamerthemeshadow} %使用 shadow 风格
%\usepackage[width=2cm,dark,tab]{beamerthemesidebar}


% 设定英文字体
\usepackage{fontspec}
\setmainfont{Times New Roman}
\setsansfont{Arial}
\setmonofont{Courier New}

% 设定中文字体
\usepackage[BoldFont,SlantFont,CJKchecksingle,CJKnumber]{xeCJK}
\setCJKmainfont[BoldFont={Adobe Heiti Std},ItalicFont={Adobe Kaiti Std}]{Adobe Song Std}

\defaultfontfeatures{Mapping=tex-text}
\usepackage{xunicode}
\usepackage{xltxtra}

\XeTeXlinebreaklocale "zh"
\XeTeXlinebreakskip = 0pt plus 1pt minus 0.1pt

\usepackage{setspace}
\usepackage{colortbl,xcolor}
\usepackage{hyperref}
%\hypersetup{xetex,bookmarksnumbered=true,bookmarksopen=true, pdfborder=1,breaklinks,colorlinks,linkcolor=blue,filecolor=black,urlcolor=cyan,citecolor=green}
\hypersetup{xetex,bookmarksnumbered=true,bookmarksopen=true, pdfborder=1,breaklinks,colorlinks,linkcolor=cyan,filecolor=black,urlcolor=blue,citecolor=green}

% 插入图片
\usepackage{graphicx}
% 指定存储图片的路径(当前目录下的figures文件夹)
\graphicspath{{figures/}}

% 可能用到的包
\usepackage{amsmath,amssymb}
\usepackage{multimedia}
\usepackage{multicol}
\usepackage{wrapfig}
\usepackage{graphicx}
\usepackage{cite}

\usepackage{indentfirst}
\setlength{\parindent}{2em}

% 定义一些自选的模板,包括背景、图标、导航条和页脚等,修改要慎重
% 设置背景渐变由10%的红变成10%的结构颜色
%\beamertemplateshadingbackground{red!10}{structure!10}
\beamertemplatesolidbackgroundcolor{white!90!blue}
% 使所有隐藏的文本完全透明、动态,而且动态的范围很小
\beamertemplatetransparentcovereddynamic
% 使itemize环境中变成小球,这是一种视觉效果
\beamertemplateballitem
% 为所有已编号的部分设置一个章节目录,并且编号显示成小球
\beamertemplatenumberedballsectiontoc
% 将每一页的要素的要素名设成加粗字体
\beamertemplateboldpartpage

% item逐步显示时,使已经出现的item、正在显示的item、将要出现的item呈现不同颜色
\def\hilite<#1>{
 \temporal<#1>{\color{gray}}{\color{blue}}
    {\color{blue!25}}
}

% 自定义彩色块状结构的颜色
\setbeamercolor{bgcolor}{fg=yellow,bg=cyan}

% 在表格、图片等得标题中显示编号
\setbeamertemplate{caption}[numbered]

% 打开PDF后直接全屏
\hypersetup{pdfpagemode={FullScreen}}

% 使用 \part,\section,\subsection 等命令组织文档结构
% 使用 \frame 命令制作幻灯片

\begin{document}

\logo{\includegraphics[height=0.09\textwidth]{bupt.png}}
\title[DASH]{On Adaptive Live Streaming\\ in Mobile Cloud Computing Environments\\ with D2D Cooperation}
\author[Zxy]{Zhang Xiaoyi}
\institute[ANT, BUPT]{Advanced Network Technology Laboratory\\ Beijing University of Posts and Telecommunications}
\date{\today}

%% 定义目录页
%\AtBeginPart{
%  \frame{
%    \frametitle{\partpage}
%    \begin{multicols}{2}
%% 如果目录过长,可以打开这个选项分两栏显示
%      \tableofcontents
%% 使用这个命令自动生成目录
%    \end{multicols}
%  }  
%}
%
% 在每个Section前都会加入的Frame
\AtBeginSection[]
{
  \begin{frame}[shrink=20]
    \frametitle{Outlines}
	\setcounter{tocdepth}{2}
    \tableofcontents[currentsection,hideallsubsections]
  %  \tableofcontents[currentsection,currentsubsection]
  \end{frame}
}
%% 在每个Subsection前都会加入的Frame
%\AtBeginSubsection[]
%{
%  \begin{frame}[shrink]
% %   \frametitle{Outlines}
%	\setcounter{tocdepth}{2}
%    %\tableofcontents[currentsubsection]
%    \tableofcontents[sectionstyle=show/shaded,subsectionstyle=show/shaded/hide]
%% 显示在目录中加亮的当前章节
%  \end{frame}
%}

\begin{frame}
  \titlepage
\end{frame}

\section{DASH}
\subsection{Background}
\begin{frame}
	\frametitle{Background}
	\begin{itemize}
		\item RTP and RTSP	
		\item	HTTP	
		\begin{itemize}
			\item	Progressive Media Downloading	
			\begin{itemize}
\item	Unstable conditions of the network may cause bandwidth waste due to reconnection or rebuffering events.
\item Does not support live streaming. (e.g., concert or football match)
\item Does not support adaptive bitrate streaming.
	\end{itemize}
\item	Dynamic Adaptive  Streaming over HTTP (DASH).	
	\begin{itemize}
\item	Splitting an original encoded video file into small pieces of selfcontained media segments.
\item Separating the media description into a single playlist file.
\item Delivering segments over HTTP.
	\end{itemize}
	\end{itemize}
	\end{itemize}
\end{frame}

\section{Cooperation - Motivation}
\subsection{Competition}
\begin{frame}
	\frametitle{Competition vs Cooperation}	
\begin{wrapfigure}{r}{0.5\textwidth}
  \vspace{-20pt}
	\begin{figure}[htbp]
    \centering
		\includegraphics[width=0.48\textwidth]{singleClient.png}
    \caption{Competitive Clients}
    \label{fig:singleClient}
    \end{figure}
%  \begin{center}
%  \end{center}
  \vspace{-20pt}
\end{wrapfigure}
The scenario: a group of {\color{red} adjacent} mobile devices are interested in watching the {\color{red} same} live streaming video.

\vspace{10pt}
When they request the video from the server separately, they may not be able to get the media content with the best streaming rate if they are under the same mobile base station: they will {\color{red} compete} for the limited network resource.
\end{frame}

\begin{frame}
	\frametitle{Competition vs Cooperation}
%\begin{wrapfigure}{l}{0.7\textwidth}
  \vspace{-20pt}
	\begin{figure}[htbp]
    \centering
		\includegraphics[width=0.75\textwidth]{xiezuo.png}
    \caption{Bandwidth Fluctuation}
    \label{fig:xiezuo}
    \end{figure}
  \vspace{-20pt}
%\end{wrapfigure}

The measured bandwidth of the three devices watching a streaming video synchronously without D2D cooperation shows the competition. The cooperated group can obtain higher video quality with more effective network resource utilization.
\end{frame}

\subsection{Cooperation}
\begin{frame}
	\frametitle{Competition vs Cooperation}	
\begin{wrapfigure}{l}{0.5\textwidth}
  \vspace{-20pt}
	\begin{figure}[htbp]
    \centering
		\includegraphics[width=0.48\textwidth]{coClient.png}
    \caption{Cooperative Clients}
    \label{fig:coclient}
    \end{figure}
%  \begin{center}
%  \end{center}
  \vspace{-20pt}
\end{wrapfigure}
The cooperation scheme takes the advantage of the {\color{red} multiple connection interfaces} on the mobile devices. 

Cellular connections can be {\color{red} extended} by wireless Device to Device (D2D) links established through WiFi-Direct or Bluetooth.
\end{frame}

\section{MCC Transcoding - Motivation}
\subsection{Transcoding}
\begin{frame}
	\frametitle{Mobile Cloud Computing (MCC) Transcoding}
  \vspace{-20pt}
	\begin{figure}[htbp]
    \centering
		\includegraphics[width=0.7\textwidth]{traDASH.png}\\
		\includegraphics[width=0.7\textwidth]{cloudDASH.png}  
    \caption{DASH with MCC}
    \label{fig:dash}
    \end{figure}
  
  \vspace{-20pt}
The media segment can be real-time transcoded by the cloud into proper rate that fits the current network bandwidth of the mobile devices.
\end{frame}

\section{System Architecture}
\subsection{Server}
\begin{frame}
	\frametitle{Server}
	\begin{figure}[htbp]
    \centering
		\includegraphics[width=0.8\textwidth]{arch.jpg}
    \caption{System Architecture}
    \label{fig:server}
    \end{figure}
\end{frame}

\begin{frame}%[shrink=10]
	\frametitle{Modules of Server}
	\begin{block}{Bandwidth Recorder}
	\scriptsize
	Recording the group bandwidth information.
	\end{block}
	\begin{block}{Media Transcoder}
	\scriptsize
	Transcoding the requested video segment into proper bitrate.
	\end{block}
	\begin{block}{HTTP Server}
	\scriptsize
	Handling the HTTP request from the devices.
	\end{block}
	\begin{block}{QoE Reward Learner}
	\scriptsize
	Evaluating the users’ experience.
	\end{block}
	\begin{block}{Rate Selection Maker}
	\scriptsize
	Making decision to select the next video bitrate.
	\end{block}
\end{frame}

\subsection{Client}
\begin{frame}
	\frametitle{Client}
	\begin{figure}[htbp]
    \centering
		\includegraphics[width=1\textwidth]{client.jpg}
    \caption{Cooperative Group of Devices}
    \label{fig:client}
    \end{figure}
\end{frame}

\begin{frame}%[shrink=20]
	\frametitle{Modules of Client}
	\begin{block}{Segment Assembler}
	\scriptsize
	Dealing with all the received video fragments. 
	\end{block}
	\begin{block}{Fragment Downloader}
	\scriptsize
	Downloading fragments from the server.
	\end{block}
	\begin{block}{Device Broadcaster}
	\scriptsize
	Sending the broadcast message to other devices.\\
	Dealing with the broadcast message from other devices.
	\end{block}
\end{frame}

\section{QoE}
\subsection{QoE}
\begin{frame}
	\frametitle{QoE Measurement}
The perceived video quality can be measured with subjective and objective approaches.
\begin{itemize}
\item Subjective - Mean Opinion Score (MOS)
\begin{itemize}
\item MOS ranges from 1 to 5, where 1 stands for the worst perceived quality, and 5 is for the best visual quality.
\item Although subjective assessment approach can directly indicate the QoE, it is impractical because of the high cost and time-consuming.
\end{itemize} 
\item Objective - Peak-Signal-to-Noise-Ratio (PSNR)
\begin{itemize}
\item PSNR computes the Mean Square Error (MSE) of the pixel between the original image and the compared one, and calculates the average value considering the number of bits occupied by each sampling point.
\item The higher PSNR means the less distortion and a video stream with higher bitrate always means the bigger PSNR.
\end{itemize} 
\end{itemize}
\end{frame}

\begin{frame}
	\frametitle{QoE Measurement}
For the DASH standard, the advantage of the buffering should be taken into consideration. 
\begin{itemize}
\item The mean segment quality level
\item The switching process of quality levels
\item The video freezes due to buffer starvations
\end{itemize}
\vspace{10pt}

The frequency of video freeze is the main factor which is responsible for the variations in the QoE. However, choosing the low bitrate video stream to fulfill the buffer is a safe selection to avoid the buffer starvation, which causes the contradiction with the desire to get video segment with high quality level.
\end{frame}

\section{Bitrate Selection}
\subsection{Bitrate Selection}
\begin{frame}
	\frametitle{Bitrate Selection}
	
	Based on PSNR
	\begin{itemize}
	\item The video stream quality level is the main factor of the QoE.
	\item Choosing the highest level of bitrate that adapts to the bandwidth limitation. 
	\item The real-time video stream transcoding mechanism makes it more bandwidth efficient. 
	\item Poor buffer.
	\end{itemize}
\end{frame}

\begin{frame}
	\frametitle{Bitrate Selection}
	
	Heuristics Algorithm - Microsoft Smooth Streaming (MSS)
	\begin{itemize}
	\item Recording the state of the device’s buffer.
	\item Realizing the steady buffer state as well as the highest video bitrate.
	\item Avoid oscillations in the decision of the bitrate selection.
	\item Being hardcoded to fit specific network condition, which is less flexible.
	\end{itemize}
\end{frame}

\begin{frame}
	\frametitle{Bitrate Selection}
	
	Q-Learning Algorithm - Mechine Learning
	\begin{itemize}
	\item Finding an optimal action-selection policy based on solving multivariate nonlinear problems.
	\item Reward value feedback from the environment will instruct the next selection of the action.
	\item Reward function can be established including the average video quality, the switch behavior and the stored buffer level.
	\item Updating the weight of the action will influence the future rate selection procedure. 
	\end{itemize}
\end{frame}

\subsection{Bitrate Selection Evaluation}
\begin{frame}
	\frametitle{Bitrate Selection Evaluation}
	\begin{figure}[htbp]
    \centering
		\includegraphics[width=0.7\textwidth]{3M.jpg}
    \caption{Bitrate Selection under Bandwidth Fluctuation}
    \label{fig:bitrate}
    \end{figure}
\end{frame}

\begin{frame}
	\frametitle{Bitrate Selection Evaluation}
	\begin{figure}[htbp]
    \centering
		\includegraphics[width=0.5\textwidth]{PSNR.jpg}
		\includegraphics[width=0.5\textwidth]{MOS.jpg}
    \caption{Evaluation using PSNR and MOS in both Cooperative and Common Modes}
    \label{fig:power}
    \end{figure}
\end{frame}

\section{Energy Efficient Bitrate Selection}
\subsection{Tradeoff}
\begin{frame}
	\frametitle{PSNR and Energy Cost}
	\begin{figure}[htbp]
    \centering
		\includegraphics[width=0.5\textwidth]{psnr_tend.jpg}
		\includegraphics[width=0.5\textwidth]{ratio_energy.jpg}
    \caption{PSNR and Energy Cost of the 29 fixed video levels}
    \label{fig:psnrenergy}
    \end{figure}
\end{frame}

\begin{frame}
	\frametitle{Tradeoff between the PSNR and the Energy Cost}
	\begin{equation}
	f(br)=c\cdot\overline{PSNR(br})-(1-c)\cdot\overline{E_{average}(br\cdot t)}
	\label{Constraint Function}
	\end{equation}
\end{frame}

\begin{frame}
	\frametitle{Influence of the Parameter c}
	\begin{figure}[htbp]
    \centering
		\includegraphics[width=0.5\textwidth]{3d.jpg}
		\includegraphics[width=0.5\textwidth]{PSNRenergyC.jpg}
    \caption{Influence of the Parameter c}
    \label{fig:c}
    \end{figure}
\end{frame}

\subsection{Evaluation}
\begin{frame}
	\frametitle{Camparision with non-Cooperative Method}
	\begin{figure}[htbp]
    \centering
		\includegraphics[width=0.5\textwidth]{compare1.jpg}
		\includegraphics[width=0.5\textwidth]{compare2.jpg}
    \caption{Bandwidth and Video Bitrate Selection of the Cooperative Scheme and the MSS Scheme}
    \label{fig:bv}
    \end{figure}
\end{frame}

\begin{frame}
	\frametitle{Compare the Energy Cost}
    \begin{table}
    \centering
    \caption{Ratio of the Energy Cost}
 %   \rowcolors[]{1}{blue!20}{blue!10}
    \begin{tabular}{|c|c|c|}
    \hline
    \rowcolor{blue!50} &MSS&$E^2Bit$\\
    \hline
    Mean & 1.0527 & 0.7832 \\
    \hline
    Min & 0.6600 & 0.5810 \\
    \hline
    Max & 1.2200 & 0.8656 \\
    \hline
    \end{tabular}
    \end{table}
\end{frame}
%\subsection{自定义}
%\begin{frame}
%	\begin{beamercolorbox}[rounded=true,shadow=true,wd=12cm]{bgcolor}
%		这是一个自定义的彩色块状结构。
%	\end{beamercolorbox}
%\end{frame}
%
%\section{分栏}
%\begin{frame}
%	\frametitle{分栏}
%	\begin{columns}
%	\column{3cm}
%	这是第一栏的文字;栏宽3cm。
%	\column{5cm}
%	这是第二栏的文字;栏宽5cm。
%	\end{columns}
%\end{frame}

\begin{frame}[plain]
	\begin{spacing}{1.5}
	\begin{center}
	\Huge{\textbf{Thanks for your attention!}}
\end{center}
\end{spacing}
\end{frame}

\end{document}
